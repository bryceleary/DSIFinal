% Options for packages loaded elsewhere
\PassOptionsToPackage{unicode}{hyperref}
\PassOptionsToPackage{hyphens}{url}
%
\documentclass[
]{article}
\usepackage{lmodern}
\usepackage{amssymb,amsmath}
\usepackage{ifxetex,ifluatex}
\ifnum 0\ifxetex 1\fi\ifluatex 1\fi=0 % if pdftex
  \usepackage[T1]{fontenc}
  \usepackage[utf8]{inputenc}
  \usepackage{textcomp} % provide euro and other symbols
\else % if luatex or xetex
  \usepackage{unicode-math}
  \defaultfontfeatures{Scale=MatchLowercase}
  \defaultfontfeatures[\rmfamily]{Ligatures=TeX,Scale=1}
\fi
% Use upquote if available, for straight quotes in verbatim environments
\IfFileExists{upquote.sty}{\usepackage{upquote}}{}
\IfFileExists{microtype.sty}{% use microtype if available
  \usepackage[]{microtype}
  \UseMicrotypeSet[protrusion]{basicmath} % disable protrusion for tt fonts
}{}
\makeatletter
\@ifundefined{KOMAClassName}{% if non-KOMA class
  \IfFileExists{parskip.sty}{%
    \usepackage{parskip}
  }{% else
    \setlength{\parindent}{0pt}
    \setlength{\parskip}{6pt plus 2pt minus 1pt}}
}{% if KOMA class
  \KOMAoptions{parskip=half}}
\makeatother
\usepackage{xcolor}
\IfFileExists{xurl.sty}{\usepackage{xurl}}{} % add URL line breaks if available
\IfFileExists{bookmark.sty}{\usepackage{bookmark}}{\usepackage{hyperref}}
\hypersetup{
  pdftitle={Food Availability in West Africa},
  pdfauthor={Bryce Leary, Milika Robbins, Zeinabou Saidou Baraze, Carine Ayidehou},
  hidelinks,
  pdfcreator={LaTeX via pandoc}}
\urlstyle{same} % disable monospaced font for URLs
\usepackage[margin=1in]{geometry}
\usepackage{graphicx,grffile}
\makeatletter
\def\maxwidth{\ifdim\Gin@nat@width>\linewidth\linewidth\else\Gin@nat@width\fi}
\def\maxheight{\ifdim\Gin@nat@height>\textheight\textheight\else\Gin@nat@height\fi}
\makeatother
% Scale images if necessary, so that they will not overflow the page
% margins by default, and it is still possible to overwrite the defaults
% using explicit options in \includegraphics[width, height, ...]{}
\setkeys{Gin}{width=\maxwidth,height=\maxheight,keepaspectratio}
% Set default figure placement to htbp
\makeatletter
\def\fps@figure{htbp}
\makeatother
\setlength{\emergencystretch}{3em} % prevent overfull lines
\providecommand{\tightlist}{%
  \setlength{\itemsep}{0pt}\setlength{\parskip}{0pt}}
\setcounter{secnumdepth}{-\maxdimen} % remove section numbering

\title{Food Availability in West Africa}
\author{Bryce Leary, Milika Robbins, Zeinabou Saidou Baraze, Carine Ayidehou}
\date{1/11/2020}

\begin{document}
\maketitle

\hypertarget{abstract}{%
\section{Abstract}\label{abstract}}

This paper analyzes how agricultural capital flows in West Africa impact
the availability of food at the country level. We examined data on food
availability in West African states from 2000 to 2017 and performed a
linear regression and a hierarchical agglomerative clustering. As the
amount of multilateral aid flows to a country increases, those in need
experience a reduction in hunger. Additionally, as a country increases
its agricultural exports the nationwide availability of food increases,
though it may not reach those experiencing hunger. From a policy
perspective, this paper argues for a focus on multilateral aid and
agricultural exports to promote food security in West Africa.

\hypertarget{introduction}{%
\section{Introduction}\label{introduction}}

According to the World Food Summit of 1996, food security is defined as
``\ldots when all people, at all times, have physical, social and
economic access to sufficient, safe and nutritious food to meet dietary
needs for a productive and healthy life'' (UN, 2014). Food insecurity is
therefore the absence of these conditions. Over decades now, the global
food security has been a major challenge for global communities to
manage. About 821 million people across the world suffered from hunger
in 2018 according to the United Nations. With the world population fast
growing, we would need to provide food for an estimate of 9 Billion
people by the year 2050 (Breene, 2016). This means that the number of
people suffering from hunger would further increase over years; as a
result, we should place more urgency on resolving this issue.

The major contributing factors to the inability to meet food security
range from a fast population growth, climate change, water scarcity, a
decrease in the number of farmers, high cost of farming (Breene, 2016)
and political instability in many countries (Maxwell, 2012).
Policymakers and global leaders are committed to ending hunger and have
codified this effort through the Millennium Challenge Goals, the
Sustainable Development Goals, and the UN's Zero Hunger Challenge.

In order to meet these goals, policy makers have invested into
initiatives to fund the agricultural sector and engaged in partnerships
with farmers and major stakeholders to bring forth comprehensive
solutions to improve the global food production. This analysis will
focus on the availability of good in the region of West Africa. We seek
to understand how capital flows in the agricultural sector influence the
availability of food in West African countries.

\hypertarget{data-collection-method}{%
\section{Data Collection Method}\label{data-collection-method}}

We are using six sets of data which originate from the Food and
Agriculture Organization of the United States (FAO). All five datasets
are time series and range from 2000 to 2017. The first dataset is the
aggregated data of Development Aid Disbursement (DevAid), which is
arranged by bilateral, multilateral and private donors across all West
African countries. The DevAid data was last updated on January 26, 2019
and was obtained by FAO through the Credit Reporting System (CRS). The
DevAid data is composed of data on the amount of aid disbursed for basic
nutrition and food aid and food security programs, measured in 2016 USD
millions.

The second dataset is the Foreign Direct Investment (FDI). which is
measured in terms of 2016 US Millions of dollars. This dataset was last
updated on November 11, 2019 and covers the total FDI inflow and
outflows to and from developing countries in the West Africa region.
This dataset was collected by the United Nations Conference on Trade and
Development (UNCTAD), The International Trade Centre (INTRACEN), the
Organization for Economic Co-operation and Development (OECD) and the
International Monetary Fund (IMF) Balance of Payments Manual. Both the
DevAid and FDI datasets are from the FAO data group of Development Flows
to Agriculture.

The third dataset is the Average Dietary Energy Supply Adequacy (ADESA)
from the FAO's Suite of Food Security Indicators which was last updated
on October 11, 2019. It is represented in a three-year average format
and is indicated as a percentage (FAO 2019). The dietary energy supply
is determined by each country's average supply of calories for food
consumption of the population.

The fourth and fifth datasets are presented as imports and exports of
crops and livestock products. Both exports and imports include the total
of aggregated agricultural products by country on an annual basis. These
datasets were last updated on October 9, 2019. This agricultural trade
data was collected corresponding to the Standard International
Merchandise Trade Statistics Methodology; the main providers of this
data are UNSD and Eurostat, but other providers are involved if needed
for non-reporting countries or missing cells (FAO 2019).

The sixth and last data set is the Depth of Food Deficit (Depth) and
originates from the World Bank Data Development Indicators, which were
sourced from the FAO's Food Security Statistics and last updated on
December 4, 2019. The key indicator in this dataset is represented in
kilocalories per person per day, based on the number of calories needed
per day to lift the undernourished population from this category when
everything else remains the same.

\begin{verbatim}
## Response depth :
## 
## Call:
## lm(formula = depth ~ bilateral + multilateral + fdi_net + exp_value + 
##     imp_value, data = agriculture)
## 
## Residuals:
##    Min     1Q Median     3Q    Max 
##  -4949  -2053   -180   1384   7662 
## 
## Coefficients:
##               Estimate Std. Error t value Pr(>|t|)    
## (Intercept)  6295.3741   351.0691  17.932  < 2e-16 ***
## bilateral     -14.9188    20.1343  -0.741  0.45980    
## multilateral  -71.0741    31.0665  -2.288  0.02345 *  
## fdi_net        -0.5423     0.2037  -2.662  0.00855 ** 
## exp_value      -0.3312     0.1592  -2.081  0.03905 *  
## imp_value      -0.3640     0.2386  -1.526  0.12902    
## ---
## Signif. codes:  0 '***' 0.001 '**' 0.01 '*' 0.05 '.' 0.1 ' ' 1
## 
## Residual standard error: 2700 on 161 degrees of freedom
##   (68 observations deleted due to missingness)
## Multiple R-squared:  0.2564, Adjusted R-squared:  0.2333 
## F-statistic:  11.1 on 5 and 161 DF,  p-value: 3.333e-09
## 
## 
## Response adesa :
## 
## Call:
## lm(formula = adesa ~ bilateral + multilateral + fdi_net + exp_value + 
##     imp_value, data = agriculture)
## 
## Residuals:
##      Min       1Q   Median       3Q      Max 
## -19.6206  -6.6908   0.1056   6.2303  24.6378 
## 
## Coefficients:
##                Estimate Std. Error t value Pr(>|t|)    
## (Intercept)   1.094e+02  1.143e+00  95.716  < 2e-16 ***
## bilateral     2.374e-01  6.556e-02   3.621 0.000392 ***
## multilateral  1.929e-01  1.012e-01   1.907 0.058322 .  
## fdi_net       3.311e-03  6.632e-04   4.992 1.54e-06 ***
## exp_value     1.660e-03  5.183e-04   3.203 0.001638 ** 
## imp_value    -1.253e-03  7.767e-04  -1.613 0.108781    
## ---
## Signif. codes:  0 '***' 0.001 '**' 0.01 '*' 0.05 '.' 0.1 ' ' 1
## 
## Residual standard error: 8.79 on 161 degrees of freedom
##   (68 observations deleted due to missingness)
## Multiple R-squared:  0.3302, Adjusted R-squared:  0.3094 
## F-statistic: 15.88 on 5 and 161 DF,  p-value: 1.061e-12
\end{verbatim}

\hypertarget{linear-regression-analysis}{%
\section{Linear Regression Analysis}\label{linear-regression-analysis}}

Prior to formally testing our hypothesis ,we executed exploratory
\texttt{ggplot}s to view the shape of our explanatory and response
variables (figure placeholder). This initial view led us to consider a
basic linear regression to fit our analysis and knowledge of statistical
methods.

This regression indicated that multilateral donor flow and export flows
are the two main influences on our response variables, \texttt{adesa}
and \texttt{depth}.

To validate this finding, we also fit the regression on a line graph
with smoothing features (figure placeholder).

Together, the regression output and graph show that as the multilateral
aid flow increases by approximately 79 million USD, the depth of the
food deficit decreases by 1,000 calories.

The same method was executed for export values and average dietary
energy supply (ADESA).

This output and graph depicts how an increase in exports reduces
available food for populations experiencing hunger across the West
African countries.

\includegraphics{dsifinal-report_files/figure-latex/regression plots-1.pdf}
\includegraphics{dsifinal-report_files/figure-latex/regression plots-2.pdf}
\includegraphics{dsifinal-report_files/figure-latex/regression plots-3.pdf}
\includegraphics{dsifinal-report_files/figure-latex/regression plots-4.pdf}
\includegraphics{dsifinal-report_files/figure-latex/regression plots-5.pdf}
\includegraphics{dsifinal-report_files/figure-latex/regression plots-6.pdf}

\hypertarget{cluster-analysis}{%
\section{Cluster Analysis}\label{cluster-analysis}}

To more closely examine the trends uncovered by the linear regression
model, we decided to complete a cluster analysis to examine how the
different states have shifted in relation to each other. This analysis
could potentially reveal additional information on how capital flows
have impacted similar states. We focused on bilateral and multilateral
development flows, as they contained the most complete data and
clustering is ineffective with significant amounts of missing data. Due
to the relatively small sample size, we utilized a hierarchical
agglomerative method clustering process as it can identify nuance in
small-n datasets which are harder to trace through k-means clustering.

This process examined the shifts in clusters at four time markers: 2000,
2005, 2010, and 2015. These time periods were selected as they contain
most of the data, avoid incomplete data, and prevent an oversaturation
of the dendrogram. The results are summarized in dendrograms created
through \texttt{hclust}, with the closeness between clusters summarized
through the \texttt{height} variable on the Y-axis.

Overall, we see countries moving further apart from each other as the
millennium progressed. This is represented by the maximum height value
moving from 4.0979373 in 2000 to 7.3082381 in 2015, a change of
-3.2103008. We can also examine Baker's Gamma correlation coefficient to
determine the level of similarity between two dendrograms. Due to the
varying size of the dendrograms, it is only possible to compare figure 3
and figure 4. Between these we find a Baker's Gamma correlation of
0.6811645. This correlation coefficient indicates a strong similarity
between the two models, though they are not exactly the same. While this
correlation coefficient doesn't account for height, we also see a change
in similarity between the two years of -1.2373249 as Niger continues to
pull away from additional countries. Thus, while we see most countries
experiencing similar effects of multinational and bilateral development
aid, their paths havd diverged over time.

\includegraphics{dsifinal-report_files/figure-latex/dendrogram plots-1.pdf}
\includegraphics{dsifinal-report_files/figure-latex/dendrogram plots-2.pdf}
\includegraphics{dsifinal-report_files/figure-latex/dendrogram plots-3.pdf}
\includegraphics{dsifinal-report_files/figure-latex/dendrogram plots-4.pdf}

\hypertarget{caveats-and-additional-notes}{%
\section{Caveats and additional
notes}\label{caveats-and-additional-notes}}

Our study includes several caveats, which we would like to address in
this section. These caveats are evaluated in terms of data collection,
data cleaning, and analysis.

\hypertarget{collection}{%
\subsection{Collection}\label{collection}}

This data was gathered by the Food and Agriculture Organization of the
United Nations (FAO). While the FAO indicates that most of its data is
from a third party which often has its own data integrity standards, our
team does not know the exact data collection methods and decisions made
in the moment. For this reason, we acknowledge that there may be unknown
sources of collection bias. To help mitigate these effects, we reviewed
the codebook for each data set.

Growth domestic product (GDP) is not included in this dataset. While GDP
is a commonly used variable when examining country-level growth , we
chose to focus on capital flows. Policymakers have several policy levers
at their disposal which can impact capital flows, such as banking
regulations or legislation. We included capital flows instead of GDP to
focus on the relationship between policymakers and nutrition outcomes.

\hypertarget{cleaning}{%
\subsection{Cleaning}\label{cleaning}}

To clean this dataset, we kept \texttt{country} and
\texttt{country\_code} as the unique identification variables for each
observation. This decision relies on the assumption that country would
be the. main variable by which we would need to use numerical
identification as fallback.

During the cleaning and merging process, (placeholder for Ivory coast)
stood out as an observation that was inconsistently spelled across the
distinct datasets. To avoid editing the excel sheet, we reconciled this
inconsistent spelling by editing the \texttt{country} column form a
factor to character. Then, we replaced the existing spelling with NA.
After this, we replaced NA with the alternative spelling.

The USD values for the Foreign Direct Investment (FDI) dataset were
adjusted from 2010 USD values to 2016 USD values. This adjustment was
made in light of the Development Aid (DevAid) dataset, which included
2016 USD values. We could have adjusted the DevAid values to 2010, but
decided to use the most recent value of 2016 instead. This logic also
considers that our dataset includes years from 2000 to 2017.

The Crops and Livestock Import and Export (Trade) datasets included
import and export values in units of thousands, while the other four
datasets of this study were in units of millions. To reconcile this
difference, we divided the trade values \texttt{exp\_value} and
\texttt{imp\_export} by 1,000 so that these values would be represented
in millions.

\hypertarget{analysis}{%
\subsection{Analysis}\label{analysis}}

Our analysis includes a linear regression and cluster analysis. While we
did attempt to. execute a Bayesian regression at the beginning of the
study, our final study does not include this analysis. We recognize that
a Bayesian regression method may have been a more robust analysis
method. However, due to our levels of statistical experience and
background, we executed a multivariate linear regression. To continue to
increase our knowledge in statistical methods, we focused on pushing our
newly gained knowledge from the course via the cluster analysis.

The linear regression in our study excludes a flow type,
\texttt{private}.The \texttt{private} observations were omitted from the
regression due to their low n-size and minimal contribution to the
exploratory Bayesian regression method mentioned above.

\hypertarget{conclusion}{%
\section{Conclusion}\label{conclusion}}

The results of this analysis for global leaders and policy makers is
clear, multilateral aid and agricultural exports increase food
availability in West Africa. More specifically, a moderate investment in
multilateral aid through organizations such as the World Bank, the World
Food Program, or the United Nations Development Program significantly
reduces the caloric deficit, helping those most in need. In turn, the
promotion of agricultural exports increases the general availability of
food in a country, supporting each facet of the population. Examining
the results of the agglomerative clustering model can help creating
multinational food security plans which build upon complementary
experiences of states, while highlight those who may be diverging from
the whole and require more nuanced programming.

Moving forward, additional research should focus on the nutritional
status of individuals and whether gains in food availability are
experienced equally across the population. It may be the case that
populations are eating enough calories, but not consuming healthy
numbers of micro and macro nutrients, or that gains in food availability
are only experienced by specific segments of the population.

\end{document}
